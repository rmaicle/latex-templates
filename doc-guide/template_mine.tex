%%%%%%%%%%%%%%%%%%%%%%%%%%%%%%%%%%%%%%%%%%%%%%%%%%



\usepackage{titlesec}
\usepackage{lipsum}



\ifpdftex
    % Taller and a bit wider than Bitstream Charter.
    % Inter-word spacing is a bit narrow.
    \usepackage{utopia}         % {put}
    % \usepackage{tgpagella}    % {qpl} looks "elegant" for technical book/report
    % \usepackage{tgbonum}      % {qbk} wide
    % \usepackage{tgtermes}     % {qtm} narrow
    % \usepackage{tgschola}     % {qcs} "darker/heavier" replacement for default serif
    %
    % Charter Note:
    % I don't like the italics specially when used as URL font.
    % It appears too slanted, specially the slash character.
    % \usepackage{charter} % {bch}

    % Sans-serif fonts
    % ----------------
    \usepackage[scaled=0.92]{helvet} % {phv}

    % Typewriter fonts
    % ----------------
    \usepackage[scaled=0.89]{beramono} % {fvm} default scaling is 0.9

    % Latin Modern Typewriter font

    % \usepackage{lmodern}               % {lmtt}

    % Make the light series be the default.
    % From: An exploration of the Latin Modern fonts
    %       by Will Robertson
    %       The PracTEX Journal
    %       TPJ2006No01, 2006-02-01
    %       http://dw.tug.org/pracjourn/2006-1/robertson/robertson.pdf

    % \DeclareFontFamily{T1}{lmtt}{}
    % \DeclareFontShape{T1}{lmtt}{m}{n}{<-> ec-lmtl10}{}
    % \DeclareFontShape{T1}{lmtt}{m}{\itdefault}{<-> ec-lmtlo10}{}
    % \DeclareFontShape{T1}{lmtt}{\bfdefault}{n}{<-> ec-lmtk10}{}
    % \DeclareFontShape{T1}{lmtt}{\bfdefault}{\itdefault}{<-> ec-lmtko10}{}

    % How to use condensed Latin Modern Typewriter font?
    % https://tex.stackexchange.com/questions/22399

    % \DeclareFontFamily{T1}{lmttc}{\hyphenchar \font-1}
    % \DeclareFontShape{T1}{lmttc}{m}{n}{<-> ec-lmtlc10}{}
    % \DeclareFontShape{T1}{lmttc}{m}{it}{<->sub*lmttc/m/sl}{}
    % \DeclareFontShape{T1}{lmttc}{m}{sl}{<-> ec-lmtlco10}{}
    % \renewcommand{\ttdefault}{lmttc}

    % zlmtt
    % -----
    % It should be placed after all your other text font loading packages
    % that might contain instructions to change \ttdefault, and before
    % loading math packages so that the math packages can make a suitable
    % definition of \mathtt.
    % From: Using zlmtt to Access the Latin Modern Typewriter Fonts
    % Michael Sharpe
    % June 11, 2019
    %
    % When used with utopia as an inline listing font, this should be
    % scaled to 1.14. As a listings font, scaling should be 1.0 or lower
    % to display reasonable number of characters per line. The light
    % condensed style may be used instead but I think it is too narrow
    % and thin on screen. I am not sure yet how it looks in print.

    \usepackage[scaled=1.0, proportional, lightcondensed]{zlmtt}

    % Make default fonts
    % ------------------
    \renewcommand*\rmdefault{put} % utopia
    \renewcommand*\sfdefault{phv}
    \renewcommand*\ttdefault{fvm}
    % Change the default font
    % https://stackoverflow.com/questions/877597/
    % \renewcommand{\familydefault}{\sfdefault}
\fi
\ifxetex
    % \defaultfontfeatures{Scale=MatchLowercase}
    \setromanfont[Scale=1.0]{IBM Plex Serif}
    \setsansfont[Scale=1.0]{Public Sans}
    % Font used in listings
    \setmonofont[Scale=0.88]{Iosevka Light}
    \setmonofont[Scale=0.88]{Iosevka}
    \setmainfont[Scale=1.0, WordSpace=1.25]{IBM Plex Serif}
    % Font used in inline texts: \lstinline and \texttt
    % See fontspec package for using \newfontfamily
    \newfontfamily\inlinettfont{Iosevka Fixed Extended}[Scale=0.94]
\fi
\ifluatex
    % Charis is a derivative of Bitstream Charter
    % Visually, Charis is a bit heavier (darker?)
    \usepackage{CharisSIL}
    \setromanfont{CharisSIL}
\fi



% Table of contents
% -----------------
\setcounter{tocdepth}{3}
\renewcommand{\cftchapterfont}{\normalsize\bfseries}
\renewcommand{\cftsectionfont}{\normalsize}
\renewcommand{\cftsubsectionfont}{\normalsize}

\setlength{\cftsectionindent}{2.1em}
\setlength{\cftsubsectionindent}{5.6em}

\setlength{\cftchapternumwidth}{2.0em}
\setlength{\cftsectionnumwidth}{3.5em}
\setlength{\cftsubsectionnumwidth}{4.9em}

\setlength{\cftbeforechapterskip}{1.0em}
\setlength{\cftbeforesectionskip}{0.1em}
\setlength{\cftbeforesubsectionskip}{0.0em}
% Narrower
\setlength{\cftsubsectionindent}{4.7em}
\setlength{\cftsectionnumwidth}{2.5em}
\setlength{\cftsubsectionnumwidth}{3.5em}



% Sentence
% -----------------------------------------------------------------------------
% Traditional typography requires a larger space to indicate the end of a
% sentence. Following along the same logic, TEX puts more space after an
% exclamation point (!), and a question mark (?). However, this tradition
% is obsolete as this extra space is disturbing. So you should almost always
% execute \frenchspacing just before the beginning of every document,
% instructing TEX to treat commas and periods in the same way.
% (Latex Manual 8.6)
\frenchspacing



%%%%%%%%%%%%%%%%%%%%%%%%%%%%%%%%%%%%%%%%%%%%%%%%%%
