\documentclass[convert={density=300,outext=.png}]{standalone}
\usepackage{tikz}
\usetikzlibrary{shapes,arrows}

% http://www.texample.net/tikz/examples/inertial-navigation-system/

\begin{document}

% We need layers to draw the block diagram
\pgfdeclarelayer{background}
\pgfdeclarelayer{foreground}
\pgfsetlayers{background,main,foreground}

\tikzstyle{block} = [rectangle, draw, fill=blue!10, text width=7.5em,
    text centered, rounded corners, minimum height=2em]
\tikzstyle{file} = [rectangle, draw, text width=7em,
    text centered, minimum height=1.5em]
\tikzstyle{line} = [draw, semithick, color=black, -latex']
\tikzstyle{cloud} = [draw, ellipse,fill=red!20, node distance=2.5cm,
    minimum height=2em]

\begin{tikzpicture}[scale=2, node distance = 2cm, auto]
    % Place nodes
    \node [block] (pdflatex) {pdflatex};
    \node [file, right of=pdflatex, node distance=4.5cm, yshift=0.4cm] (texfiles) {tex};
    \node [file, right of=pdflatex, node distance=4.5cm, yshift=-0.4cm] (imagefiles) {image};
    \node [block, below of=pdflatex, node distance=2cm] (pp) {preprocessor};
    \node [file, right of=pp, node distance=4.5cm, yshift=0.4cm] (markdownfiles) {markdown};
    \node [file, right of=pp, node distance=4.5cm, yshift=-0.4cm] (ppfiles) {pp};
    \node [block, below of=pp, node distance=2cm] (pandoc) {pandoc};
    \node [file, right of=pandoc, node distance=4.5cm, yshift=0cm] (templatefile) {template};
    \node [file, below of=pandoc, node distance=2cm] (pdffile) {pdf file};

    %\path (pdflatex.140)+(2.7,0) node (texfiles) [file] {tex files};
    %\path (pdflatex.-140)+(2.7,0) node (imagefiles) [file] {image files};
    %\path (pandoc.0)+(1.44,0) node (markdownfiles) [file] {markdown files};

    %\path [line,dashed] (texfiles){tex files} -- node [color=black] {input} (pdflatex);

    %\node [file, right of=pdflatex, node distance=4.5cm] (texfiles) {tex files};
    %\node [file, right of=pandoc, node distance=4.5cm] (markdownfiles) {markdown files};


    %\node [file, anchor=north west, node distance=4cm] at (init.east) (markdown) {markdown files};
    %\node [file, anchor=south west, node distance=4cm] at (init.east) (images) {image files};

    %\path (init.140)+(2.5,0) node (markdown) [file] {markdown files};
    %\path (init.-140)+(2.5,0) node (images) [file] {image files};

    %\node [cloud, right of=init] (system) {system};
    %\node [block, below of=init] (identify) {identify candidate models};
    %\node [block, below of=identify] (evaluate) {evaluate candidate models};
    %\node [block, left of=evaluate, node distance=4.5cm] (update) {update model};
    %\node [block, below of=decide, node distance=2.5cm] (stop) {stop};
    % Draw edges
    \path [line] (pdflatex) -- (pp);
    \path [line] (pp) -- (pandoc);
    \path [line,dashed] (pp) -- (pandoc);
    %\path (pandoc) edge[bend left=75, ->, dotted, semithick, -latex'] node [right, color=black] {calls} (pdflatex);

    %\path [line] (evaluate) -- (decide);
    %\path [line] (decide) -| node [near start, color=black] {yes} (update);
    %\path [line] (update) |- (identify);
    %\path [line] (decide) -- node [, color=black] {no}(stop);

    %\path [line, dashed] (texfiles) -- (pdflatex);
    %\path [line, dashed] (pdflatex) -- (imagefiles);
    %\path [line, dashed] (markdownfiles) -- (pp);
    %\path [line, dashed] (pp) -- (ppfiles);
    \path (imagefiles) edge[->, dashed, semithick, -latex'] node {} (markdownfiles);

    \path (texfiles) edge[bend right=8, ->, dashed, semithick, -latex'] node {} (pdflatex);
    \path (pdflatex) edge[bend right=8, ->, dashed, semithick, -latex'] node {} (imagefiles);
    \path (markdownfiles) edge[bend right=8, ->, dashed, semithick, -latex'] node {} (pp);
    \path (pp) edge[bend right=8, ->, dashed, semithick, -latex'] node {} (ppfiles);
    \path (ppfiles) edge[bend right=8, ->, dashed, semithick, -latex'] node {} (pandoc);
    \path (templatefile) edge[bend left=8, ->, dashed, semithick, -latex'] node {} (pandoc);


    \path [line,dashed] (pandoc) -- node [color=black, right, yshift=-0.3cm] {output} (pdffile);
    %\path [line,dashed] (imagefiles) -- (pandoc);
    %\path (pandoc) edge[bend right=50, ->, dotted, thick, -latex'] node [right, color=black] {calls} (pdflatex);


    %\path [line,dashed] (system) -- (init);
    %\path [line,dashed] (system) |- (evaluate);

    \path (pdflatex.north)+(0.0,0.3) node (buildscript) {build script};

    % Now it's time to draw the colored IMU and INS rectangles.
    % To draw them behind the blocks we use pgf layers. This way we
    % can use the above block coordinates to place the backgrounds
    \begin{pgfonlayer}{background}
        % Compute a few helper coordinates
        %\path (pdflatex.west |- texfiles.north)+(-0.3,0.1) node (a) {};
        %\path (INS.south -| pandoc.east)+(0.1,-0.1) node (b) {};

        \path (pdflatex.north west)+(-0.3,0.5) node (a) {};
        \path (pandoc.south east)+(0.3,-0.3) node (b) {};
        \path[fill=yellow!20,rounded corners, draw=black!50, dashed]
            (a) rectangle (b);
        %\path (gyros.north west)+(-0.2,0.2) node (a) {};
        %\path (IMU.south -| gyros.east)+(+0.2,-0.2) node (b) {};
        %\path[fill=blue!10,rounded corners, draw=black!50, dashed]
        %    (a) rectangle (b);
    \end{pgfonlayer}
\end{tikzpicture}
\end{document}
